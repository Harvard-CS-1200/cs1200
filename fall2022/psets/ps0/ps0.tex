\documentclass[11pt]{article}
\usepackage{classTools}
\usepackage{graphicx}
\def\draft{1}

\begin{document}
\psHeader{0}{Wed Sep. 7, 2022 (11:59PM)}

The purpose of this problem set is to reactivate your skills in proofs and programming from CS20 and CS32/CS50. For those of you who haven't taken one or both those courses, the problem set can also help you assess whether you have acquired sufficient skills to enter CS120 in other ways and can fill in any missing gaps through self-study. Even for students with all of the recommended background, this problem set may still require a significant amount of thought and effort, so do not be discouraged if that is the case and do take advantage of the staff support in section and office hours. 

For those of you who are wondering whether you should wait and take CS20 before taking CS120, we encourage you to also complete  \href{https://drive.google.com/file/d/1QIJR6sb9hfkK67PhpQaK9KQBzYwzXvsW/view}{the CS20 Placement Self-Assessment}.  Some problems there that are of particular relevance to CS120 and are complementary to what is covered below are Problems 2 (counting), 4 (comparing growth rates), 9 (quantificational logic), and 12 (graph theory). 

Written answers must be submitted in pdf format on Gradescope. Although \LaTeX{} is not required, it is strongly encouraged. You may handwrite solutions so long as they are fully legible. The \texttt{ps0} directory, which contains your code for problems 1a and 1c, must be submitted separately to an autograder on Gradescope. Be sure to pull the starter code from the \href{https://github.com/Harvard-CS-120/cs120}{cs120 GitHub repository}.

 \newcommand{\children}{\mathit{children}}
 %\renewcommand{\treeroot}{\mathit{treeroot}}
 \newcommand{\parent}{\mathit{parent}}
 
\begin{enumerate}
\item (Binary Trees) 
\iffalse 
Recall the following properties of a (rooted) {\em tree}:
    \begin{itemize}
        \item A (rooted) {\em tree} $T$ consists of a finite set $V$ of vertices (also sometimes called ``nodes''), one of which is the {\em root} $r\in V$
        \item Every vertex $v$ has a finite set $\children(v)\subseteq V-\{r\}$ of children; no two vertices $v\neq w$ share any children (i.e. $\children(v)\cap \children(w) = \emptyset$) and every vertex other than the root is the child of some vertex (i.e. $\bigcup_{v\in V} \children(v) = V-\{r\}$)
        \item For a vertex $v$ other than $r$, $\parent(v)$ is the unique vertex $w$ such that $v\in \children(w)$.
        \item A {\em leaf} of a (rooted) tree is a vertex $v$ with no children.
        \item  The {\em size} of a tree is its number of vertices.
        \item A vertex $w$ is a {\em descendent} of a vertex $v$ if there is a sequence of vertices $v_0,v_1,\ldots,v_k\in V$, $k\in \mathbb{N}$ such that $v_0=v$, $v_k=w$, and $v_i\in \children(v_{i-1})$ for $i=1,\ldots,k$.
        \footnote{$\mathbb{N}$ denotes the natural numbers $\{0,1,2,3,\ldots\}$.  Since we are computer scientists, we start counting at 0.}
        \item The sequence $(v_0,v_1,\ldots,v_k)$ is called a {\em path} from $v$ to $w$ and $k$ is the {\em distance} from $v$ to $w$.
 Taking $k=0$, we see that $v$ is a descendent of itself.
        \item Given any vertex $v$ in a tree, the {\em subtree} rooted at $v$ is the tree consisting of all of $v$'s descendents, with $v$ as the root.
        \item The {\em height} of a tree $T$ is the largest distance from the root to a leaf.
    \end{itemize}
    
 A binary tree has the additional constraint where every vertex $v$ has 0, 1, or 2 children.
\fi
 In the \href{https://github.com/Harvard-CS-120/cs120}{cs120 GitHub repository}, we have given you a Python implementation of a binary tree data structure, as well as a collection of test trees built using this data structure.  We specify a binary tree by giving a pointer to its {\em root}, which is a special {\em vertex} (a.k.a. {\em node}), and giving every vertex pointers to its {\em children} vertices and its {\em parent} vertex as well as an identifying {\em key}: 
 
 \begin{verbatim}
    class BinaryTree:
        def __init__(self, root):
            self.root: BTvertex = root
 
    class BTvertex:
        def __init__(self, key):
            self.parent: BTvertex = None
            self.left: BTvertex = None
            self.right: BTvertex = None
            self.key: int = key
            self.size: int = None
 \end{verbatim}


 In CS50, the concept of a Python \texttt{class} was not covered. Here, with \texttt{BinaryTree} and \texttt{BTvertex}, we are using them in the same way as a \texttt{struct} in C. An object \btv\ of the \texttt{BTvertex} class contains five attributes, which we list with the type of the object we expect to be named by each attribute (using the Python type annotation syntax). These attributes can be accessed as \texttt{v.parent}, \texttt{v.left}, \texttt{v.right}, \texttt{v.key}, and \texttt{v.size}. 
 For example, \texttt{v.left.key} is the key associated with \btv's left child. An object of the \texttt{BinaryTree} class contains only one attribute, which is the \texttt{BTvertex} object that is the root of our binary tree. You can create a \texttt{BinaryTree} object as follows:
 
\begin{verbatim}
root = BTvertex(120)
tree = BinaryTree(root)
tree.root.left = BTvertex(121)
tree.root.right = BTvertex(124)
\end{verbatim}

You can then print attributes of the newly created \texttt{BinaryTree} object:
\begin{verbatim}
print(tree.root.key)
>> 120
print(tree.root.left.key)
>> 121
\end{verbatim}
 

 Classes are more general than structs because they can also have private attributes and methods that operate on the attributes, allowing for object-oriented programming. However, you won't need that generality in this problem set.

 Here is an instance \treeT\ of \texttt{BinaryTree}:
 
 \includegraphics[scale=.175]{ps0_assets/p0_q1_BT_before.png}

 A \texttt{BinaryTree} \treeT\  contains only a pointer to its root vertex, \texttt{T.root}, which is required to satisfy \texttt{T.root.parent==None}. In the above example, 
 the root is the vertex with key 1 (i.e. \texttt{T.root.key==1}).
 A binary tree vertex \btv\ can have zero, one, or two children, determined by which of \texttt{v.left} and  \texttt{v.right} are equal to \texttt{None}.    In the above example, the vertex \btv\ with key 3 has 
 \texttt{v.left==None} but \texttt{v.right} is the vertex with key 6.
 A {\em leaf} is a vertex with zero children, i.e. \texttt{v.left==v.right==None}. 
 
 A vertex \btw\ is {\em descendent} of a vertex \btv\ if there is a sequence of vertices $\btv_0,\btv_1,\ldots,\btv_k$, $k\in \mathbb{N}$ such that $\btv_0=\btv$, $\btv_k=\btw$, and 
 $\btv_i \in \{\btv_{i-1}.\texttt{left},\btv_{i-1}.\texttt{right}\}$ for $i=1,\ldots,k$.\footnote{$\mathbb{N}$ denotes the natural numbers $\{0,1,2,3,\ldots\}$.  Since we are computer scientists, we start counting at 0.}
 In the above example, the vertex with key 5 is a descendent of the root (with a path of length 2), but is not a descendent of the vertex with key 3.
 The sequence $\btv_0,\btv_1,\ldots,\btv_k$ is called a {\em path} from \btv\ to \btw\ and $k$ is the {\em distance} from \btv\ to \btw. Taking $k=0$, we see that \btv\ is a descendent of itself.

 The {\em vertex set} of a binary tree \treeT\ consists of all of the descendents of \texttt{T.root}. The {\em size} of \treeT\ is its number of vertices. The {\em height} of \treeT\ is the largest distance from the root to a leaf.  The above example has size 8 and height 3.
 
 Given any vertex \btv\ in a tree, the {\em subtree} rooted at \btv\ consists of all of \btv's descendents.  Note that we can remove a subtree and turn it into a new tree \texttt{S} by setting
 \texttt{S.root=v} and \texttt{v.parent=None}.

 For now, the \texttt{key} attribute serves to distinguish vertices from each other in our tests and help illustrate what the algorithms are doing.  The \texttt{BTvertex}\ class
 also has a \texttt{size} attribute, which is initialized to \texttt{None} in all of the test instances; it will be filled in by the program you write in Part~\ref{part:calculatesizes}.

 An instance \treeT\ \texttt{BinaryTree} is {\em valid} if it satisfies the following constraints: \begin{itemize}
     \item \texttt{T.root.parent==None}
     \item \treeT\ has finitely many vertices.
     \item No two vertices \btv, \btw\ of \treeT\ share a child, i.e. 
     $\{\texttt{v.left},\texttt{v.right}\} \cap \{\texttt{w.left},\texttt{w.right}\} = \emptyset$. 
 \end{itemize}
 All of the test instances we provide are valid, and furthermore have the property that all of the vertices have distinct keys (which is something we often want, but not always).

 \begin{enumerate}
 \item \label{part:calculatesizes} (recursive programming)
 Write a recursive program \texttt{calculate\_sizes} that given a vertex \btv\ of a binary tree \treeT, calculates the sizes of all of the subtrees rooted at descendents of \btv.  After running your program on \texttt{T.root}, every vertex \btv\ in \treeT\ should have \texttt{v.size} set to the size of the subtree rooted at \btv. (Recall that the size attributes are initialized to \texttt{None}.)  We call the resulting tree a {\em size-augmented} tree.
 
For example, if \treeT\  is the  tree shown above, 
then calling \texttt{calculate\_sizes(T.root)} should modify  \treeT\ to be the following size-augmented tree:

 \includegraphics[scale=.175]{ps0_assets/p0_q1_BT_after.png}

 Your program should run in time $O(n)$ when given the root of a tree with $n$ vertices. In a sentence or two, informally justify why your program has such a runtime. \\
 
 SOLUTION: The program has a runtime of $O(n)$ because we visit each vertex once to calculate the size of the subtree rooted at that vertex. With the recursize structure, when we make a call to some vertex, a subsequent call is made on the child vertices, until no children remain, such function calls referring to vertices in earlier layers remain on the call stack while the sizes of trees rooted at lower level vertices are calculated first. 
 
 \item (proofs by contradiction) \label{part:contradiction}
 Removing a vertex \btv\ from a tree \treeT\ yields up to three disjoint trees: the subtree rooted at
 \texttt{v.left} (unless \texttt{v.left==None}), the subtree rooted at
 \texttt{v.right} (unless \texttt{v.right==None}), and a tree rooted at \texttt{T.root} consisting of all non-descendants of \btv\ (unless \texttt{T.root==v}).  For example:
 \\

 Before:
 
 \includegraphics[scale=.175]{ps0_assets/p0_q1b_before.png}
 
 
 After:
 
  \includegraphics[scale=.225]{ps0_assets/p0_q1b_after.png}

  Prove that in every tree \treeT\ of size $n$, there exists a vertex \btv\ such that removing \btv\ from \treeT\ results in disjoint trees that all have size at most $n/2$.  \\
  
  You may prove this however you like, but a recommended approach is to define a ``potential function'' $\phi$ on the vertices of the tree, by setting $\phi(\btv)$ to equal the size of the largest tree created by removing \btv.  Let $\btv^*$ be a vertex that minimizes the value of $\phi$, i.e. $\btv^*$ is a vertex such that $\phi(\btv^*) \le \phi(\btv)$ for all other vertices $\btv$. Then we want to prove that $\phi(\btv^*)\leq n/2$.  Prove this by contradiction.  (Hint: try to show that either the parent or one of the children will have smaller potential. If you're feeling stuck, try drawing some pictures!) \\
  
  SOLUTION: We define as suggested a vertex $\btv^*$ that minimizes the value of $\phi$, i.e. $\btv^*$ is a vertex such that $\phi(\btv^*) \le \phi(\btv)$ for all other vertices $\btv$, where $\phi$ represents a ``potential function`` on the tree such that $\phi(\btv)$ is equal to the size of the largest tree created by removing \btv. We wish to prove that  $\phi(\btv^*)\leq n/2$. We prove this by contradiction. Suppose that $\phi(\btv^*) > n/2$, that is, the largest disjoint tree obtained by removing $\btv^*$ has size $ > n/2$. Given that there are a total $n$ vertices, and the vertex $\btv^*$ is one of them, this places a bound on the size of the remaining (at most) two disjoint trees obtained when we removed $\btv^*$, where the combined size is $ < \lfloor(n/2)\rfloor - 1$, again because we already had $\phi(\btv^*) > n/2$. \\
  
  Now, consider the vertex obtained by traveling one step in the direction of the largest disjoint tree obtained by removing $\btv^*$, regardless of whether this is the parent or a child. We denote this vertex $\btv^{**}$. Then, since this vertex is contained within the largest disjoint tree obtained by removing $\btv^*$ (the disjoint tree which $\phi(\btv^*)$ refers to), the potential in the same direction of this disjoint tree, obtained by removing $\btv^{**}$ will be $\phi(\btv^*) - 1 < \phi(\btv^*)$. Meanwhile, in the other direction, the total size of the other (at most) 2 disjoint subtrees would be $ < \lfloor(n/2)\rfloor - 1 + 1$, or $ < \lfloor(n/2)\rfloor$, which is still less than $\phi(\btv^*) > n/2$. Since , all disjoint trees of $\btv^{**}$ have size less than $\phi(\btv^*)$, we have that $\phi(\btv^{**}) < \phi(\btv^*)$, contradicting the initial assumption that $\phi(\btv^*) \le \phi(\btv)$ for all other vertices $\btv$. Thus, by contradiction, we have shown it must be true that $\phi(\btv^*)\leq n/2$, and there must be some vertex such that removing that vertex would result in disjoint trees with size at most $n/2$. \\
 
 \item (from proofs to algorithms)  Turn your proof from Part~\ref{part:contradiction} into a Python program that given a root vertex \texttt{r} of a {\em size-augmented} tree \treeT\ with $n$ vertices finds a vertex \btv\ with $\phi(\btv)\leq n/2$. Your program should run in time $O(h)$ on all size-augmented trees of height $h$; again informally justify why your program has such a runtime. (Hint: try to repeatedly reduce the potential function by moving to children. Why don't we need to try moving to parents as in the previous proof?) \\
 
 EXLANATION: The program has a runtime of $O(h)$ because at each vertex of the tree, it only checks at most two vertices in the subsequent layer, its left child, and right child. This is achieved within the bounds of the while loop. In this loop, the vertex being checks updates to one of these two children vertices for the next iteration unless the desired conditions are met. Thus, despite the size of the tree, at each layer we are essentially checking conditions for only 2 vertices, and thus the runtime is constant with respect to the number of vertices in a given layer, resulting in a runtime of $O(h)$. \\
 
 Furthermore, we don't have to check the parents because of the nature of iterating through the tree starting at the root vertex. This results in starting with the largest possible subtree, so subsequent subtree lengths would automatically be smaller and decrease in size as we travel through the tree at each iteration of the while loop. 
 \end{enumerate}
 
 \newcommand{\incomp}{\mathit{incomp}}
 \item (matchings and induction)
 Later in the course, we will study matching algorithms that are used in practice to match kidney donors to patients.  The challenge in general is that some donors are incompatible with some patients (i.e. the patient's body is likely to reject the donated kidney).  Suppose we are very lucky and have $n$ donors and $n$ patients where each donor $d$ is incompatible with exactly one patient, denoted $\incomp(d)$, and each patient $p$ is incompatible with exactly one donor $\incomp(p)$. (Incompatibility is symmetric, so $\incomp(d)=p$ iff $\incomp(p)=d$.)  Let $f(n)$ be the number of ways, under these circumstances, of matching donors to patients so that each donor donates exactly one kidney to a compatible patient and each patient receives exactly one kidney from a compatible donor.  
 \begin{enumerate} 
\item Show that $f(1)=0$, $f(2)=1$, and for all $n\geq 3$, we have
 $$ f(n) = (n-1)\cdot (f(n-1)+f(n-2)).$$
 (Hint: let $d$ be one of the donors, and consider all possible patients $p$ with whom $d$ could be matched.  Then consider cases according to whether $\incomp(p)$ is matched with $\incomp(d)$ or not.) \\
 
 SOLUTION: First we prove the statement when $n=1$ and $n=2$. When $n=1$, the only donor is incompatible with the only patient, so there are no ways to match them, meaning $f(1) = 0$. Additionally, when $n=2$, suppose we have pairs $(d_1, p_1)$ and $(d_2, p_2)$ where $p_i = incomp(d_i)$ and vice versa. Then the only way to match the donors and patients successfully is by assigning $d_1$ to $p_2$ and $d_2$ to $p_1$. Thus, we have proved $f(2) = 1$. 
 
 We will mark each donor and patient by $d_1, d_2, \dots, d_n$ and $p_1, p_2, \dots, p_n$ where $d_i$ is incompatible with $p_i$, that is, $p_i = incomp(d_i)$ and vice versa. Suppose we first start assigning some donor $d_j$. We then have $(n-1)$ ways of doing this, since we can match $d_j$ with all $p_1 \dots p_n$ except $p_j$. Suppose $d_j$ ends up being assigned to some patient $p_k$. Then, we consider how the rest of the assignments will be done by splitting up which patient $incomp(p_k) = d_k$ is assigned to. \\
 
 First, consider the case in which $d_k$ is assigned to $incomp(d_j) = p_j$. Then, we have essentially matched $d_j$ and $d_k$ to the other's patient, eliminating these two donor, patient pairs. There are then only $n-2$ donor, patient pairs remaining, which none of have been matched. Thus, we total number of ways to match these remaining $n-2$ donor, patient pairs is $f(n-2)$. \\
 
 Now, for the second case. We must consider all the remaining ways in which $d_k$ is not matched with $incomp(d_j) = p_j$, since we already counted those. Thus, essentially, we can treat $p_j$ as $incomp(d_k)$!. Since the patient corresponding to $d_k$, $p_k$ has already been matched, $p_j$ essentially replaces $p_k$. In this case, we are treating the fact that we only want to count the ways of assigning where $d_k$ is not matched with $p_j$ as the same as $d_k$ being incompatible with $p_j$, since we already counted these ways in the previous case. Thus, we have matched one pair and still have $n-1$ remaining pairs to match, where each donor is incompatible with one patient (or can't be matched). The total number ways of matching in this case is $f(n-1)$. Thus, the total number of ways for matching $n$ donors and patients is as described, 
 
  $$ f(n) = (n-1)\cdot (f(n-1)+f(n-2)).$$
 

 \item Prove by strong induction that for all $n\geq 2$, we have
 $$\frac{n!}{3} \leq f(n) \leq \frac{n!}{2}.$$
 
 First we prove base cases. Suppose $n=2$. We already know from part $a$ that $f(2) = 1$. Then, by plugging into the inequality, we must check that 
$$
\frac{2!}{3} \leq f(2) \leq \frac{2!}{2}
$$
$$
 = \frac{2}{3} \leq 1 \leq \frac{2}{2}
$$
$$
 = \frac{2}{3} \leq 1 \leq 1
$$

which is indeed true. We now move on to check the next base case, which is when $n=3$. We must check

$$
\frac{3!}{3} \leq f(3) \leq \frac{3!}{2}
$$

Simplifying, we have 
$$
\frac{6}{3} \leq (3-1) \times \left( f(2) + f(1) \right) \leq \frac{6}{2}
$$

$$
= 2 \leq 2 \times (1 + 0) \leq 3
$$

$$
= 2 \leq 2 \leq 3
$$

which is indeed true. 

We now move on to the inductive proof. Suppose the statement is true for both $k$ and $k+1$. That is, 

 $$\frac{k!}{3} \leq f(k) \leq \frac{k!}{2}.$$
 
 $$\frac{(k+1)!}{3} \leq f(k+1) \leq \frac{(k+1)!}{2}.$$
 
 Then, by adding respective portions of the inequality, we have that 
 
 $$
= \frac{k!}{3} + \frac{(k+1)!}{3} \leq f(k) + f(k+1) \leq \frac{k!}{2} + \frac{(k+1)!}{2}
 $$
 $$
= \frac{k! \times (1 + k + 1)}{3} \leq f(k) + f(k+1) \leq \frac{k! \times (1 + k + 1)}{2}
 $$
 $$
= \frac{k! \times (k+2)}{3} \leq f(k) + f(k+1) \leq \frac{k! \times (k+2)}{2}
 $$
 Then, we multiply everything by a factor $(k+1)$. 
 $$
\frac{k! \times (k+1) \times (k+2)}{3} \leq (k+1) \times \left(f(k) + f(k+1)\right) \leq \frac{k! \times (k+1) \times (k+2)}{2}
 $$
 
$$
 = \frac{(k+2)!}{3} \leq (k + 2 - 1) \times  \left(f(k) + f(k+1)\right) \leq \frac{(k+2)!}{2}
$$

Thus, we have proved this holds for $k+2$, and thus we statement is true for all $n \geq 2$.
 
 
 
 \end{enumerate}

\end{enumerate}


\end{document}

\iffalse